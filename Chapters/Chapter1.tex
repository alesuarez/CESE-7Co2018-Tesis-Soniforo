% Chapter 1

\chapter{Introducción General} % Main chapter title

\label{Chapter1} % For referencing the chapter elsewhere, use \ref{Chapter1} 
\label{IntroGeneral}

%----------------------------------------------------------------------------------------

% Define some commands to keep the formatting separated from the content 
\newcommand{\keyword}[1]{\textbf{#1}}
\newcommand{\tabhead}[1]{\textbf{#1}}
\newcommand{\code}[1]{\texttt{#1}}
\newcommand{\file}[1]{\texttt{\bfseries#1}}
\newcommand{\option}[1]{\texttt{\itshape#1}}
\newcommand{\grados}{$^{\circ}$}

%----------------------------------------------------------------------------------------

%\section{Introducción}

%----------------------------------------------------------------------------------------

Este capítulo introduce al lector al tema abordado en este trabajo, su propósito y su alcance.

\section{Motivacion}

En Argentina 1 de cada 10 personas poseen algún tipo de discapacidad, siendo la que más prevalece la discapacidad motora, seguida por la visual, la auditiva y la mental \cite{estudioIndec}.

En este marco nos enfocamos en las personas con discapacidad visual, las cuales se enfrentan diariamente con el desafío de calcular distancias en un lugar determinado, con el fin de evitar accidentes. El uso del bastón es el elemento empleado a la hora de desplazarse para evitar dicha problemática. Bajo este contexto el mayor inconveniente surge en la vía publica al querer cruzar una calle.  Esto se debe a que en general no tienen ningún tipo de señal que les indique si está libre de autos circulando o si se produjo el cambio de semáforo. Para estas situaciones el bastón no es de utilidad, ya que no brinda ningún tipo de información. Si bien en algunas oportunidades pueden valerse de la buena intención de algún transeúnte es importante para toda persona poder valerse por sí misma y no depender de un tercero. 

\section{Objetivos}

El objetivo de este proyecto fue desarrollar un prototipo abierto, autónomo y económico, que permita ser conectado a cualquier semáforo convencional, aprendiendo su comportamiento. Tomando como entradas la secuencia de luces del semáforo y como salida advierta a las personas con discapacidades visuales
cuando puede o no cruzar la calle sin peligro. Esta advertencia se realiza por medio de de una señal sonora y/o vibraciones de su teléfono inteligente el cual se conecta a una red wifi \citep{raeWifi} provista por el sistema.

Dicho prototipo fue realizado de manera abierta, es decir, que su código y hardware están disponibles para todas las personas que lo  deseen.

%----------------------------------------------------------------------------------------

\section{Alcance}

El desarrollo del presente proyecto incluyó:
\begin{itemize}
\item Desarrollo de un prototipo funcional.
\item Desarrollo de una aplicación en android.
\item Desarrollo de un protocolo para lograr la escalabilidad en las formas de comunicación con dispositivos de advertencia.
\item Ajuste de nivel de sonido automáticamente según ruido ambiente.
\item Posibilidad de activar el sistema por medio de un comando a distancia.
\end{itemize}

%----------------------------------------------------------------------------------------






